\xrtag{Structural}{Structural 创建型}
\chapter{Structural 结构型}

\textbf{Also known as 别名}

包装器 Wrapper


Structural patterns are concerned with how classes and objects are composed to form larger structures. Structural class patterns use inheritance to compose interfaces or implementations.

结构型模式涉及到如何组合类和对象以获得更大的结构。结构型类模式采用继承机制来组合接口或实现。

一个简单的例子是采用多重继承方式将两个以上的类组合成一个类,结果这个类包含了所有父类的性质。这一模式尤其有助于多个独立开发的类库协同工作。另外一个例子是类型是的Adapter模式。一般来说,适配器使得一个接口与其它接口兼容,从而给出了多个不同接口的统一抽象。为此,类适配器对一个Adaptee类进行私有继承。这样,适配器就可以用Adaptee的接口表示他的接口。

结构型对象模式不是对接口和实现进行组合,而是描述了如何对一些对象进行组合,从而实现新功能的一些方法。因为可以在运行时刻改变对象组合关系,所以对象组合方式具有更大的灵活性,而这种机制用静态类组合是不可能实现的。


\section{Adapter 适配器}

\subsection{Structure 结构}

A class adapter uses multiple inheritance to adapt one interface to another:

类适配器使用多重继承对一个接口与另一个接口进行匹配,如下图所示:

\htmlpic{images/Adapter_MultiInheritance.png}

An object adapter relies on object composition:

对象匹配器依赖于对象组合,如下图所示:

\htmlpic{images/Adapter_ObjComposition.png}

\subsection{Participants 组成}

\begin{itemize}

\item \textbf{Target} (Shape)

	\begin{itemize}
		\item defines the domain-specific interface that Client uses.
	\end{itemize}

\item \textbf{Client} (DrawingEditor)

	\begin{itemize}
		\item collaborates with objects conforming to the Target interface.
	\end{itemize}

\item \textbf{Adaptee} (TextView)

	\begin{itemize}
		\item defines an existing interface that needs adapting.
	\end{itemize}

\item \textbf{Adapter} (TextShape)

	\begin{itemize}
		\item adapts the interface of Adaptee to the Target interface.
	\end{itemize}

\end{itemize}

\subsection{Collaborations 协作}

\begin{itemize}
	\item Clients call operations on an Adapter instance. In turn, the adapter calls Adaptee operations that carry out the request.

	客户在Adapter实例上调用一些操作。接着适配器调用Adaptee的操作实现这个请求。
\end{itemize}

