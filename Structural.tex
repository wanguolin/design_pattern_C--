\xrtag{Structural}{Structural 创建型}
\chapter{Structural 结构型}

\textbf{Also known as 别名}

包装器 Wrapper


Structural patterns are concerned with how classes and objects are composed to form larger structures. Structural class patterns use inheritance to compose interfaces or implementations.

结构型模式涉及到如何组合类和对象以获得更大的结构。结构型类模式采用继承机制来组合接口或实现。

一个简单的例子是采用多重继承方式将两个以上的类组合成一个类,结果这个类包含了所有父类的性质。这一模式尤其有助于多个独立开发的类库协同工作。另外一个例子是类型是的Adapter模式。一般来说,适配器使得一个接口与其它接口兼容,从而给出了多个不同接口的统一抽象。为此,类适配器对一个Adaptee类进行私有继承。这样,适配器就可以用Adaptee的接口表示他的接口。

结构型对象模式不是对接口和实现进行组合,而是描述了如何对一些对象进行组合,从而实现新功能的一些方法。因为可以在运行时刻改变对象组合关系,所以对象组合方式具有更大的灵活性,而这种机制用静态类组合是不可能实现的。


\section{Adapter 适配器}

\subsection{Structure 结构}

A class adapter uses multiple inheritance to adapt one interface to another:

类适配器使用多重继承对一个接口与另一个接口进行匹配,如下图所示:

\htmlpic{images/Adapter_MultiInheritance.png}

An object adapter relies on object composition:

对象匹配器依赖于对象组合,如下图所示:

\htmlpic{images/Adapter_ObjComposition.png}

\subsection{Participants 组成}

\begin{itemize}

\item \textbf{Target} (Shape)

	\begin{itemize}
		\item defines the domain-specific interface that Client uses.
	\end{itemize}

\item \textbf{Client} (DrawingEditor)

	\begin{itemize}
		\item collaborates with objects conforming to the Target interface.
	\end{itemize}

\item \textbf{Adaptee} (TextView)

	\begin{itemize}
		\item defines an existing interface that needs adapting.
	\end{itemize}

\item \textbf{Adapter} (TextShape)

	\begin{itemize}
		\item adapts the interface of Adaptee to the Target interface.
	\end{itemize}

\end{itemize}

\subsection{Collaborations 协作}

\begin{itemize}
	\item Clients call operations on an Adapter instance. In turn, the adapter calls Adaptee operations that carry out the request.

	客户在Adapter实例上调用一些操作。接着适配器调用Adaptee的操作实现这个请求。
\end{itemize}

\subsection{Consequences GoF评论}

类适配器和对象适配器有不同的权衡,类适配器:

\begin{itemize}

	\item 用一个具体的Adapter类对Adaptee和Target进行匹配。结果是当我们想要匹配一个类以及所有它的子类时,类Adapter将不能胜任工作。

	\item 使得Adapter可以重定义Adaptee的部分行为,因为Adapter是Adaptee的一个子类。

	\item 仅仅引入了一个对象,并不需要额外的指针以间接得到Adaptee。

\end{itemize}

对象适配器则:

\begin{itemize}

	\item 允许一个Adapter与多个Adaptee——即Adaptee本身以及它的所有子类(如果有子类的话)——同时工作。Adapter也可以一次给所有的Adaptee添加功能。

	\item 使得重定义Adaptee的行为比较困难。这就需要生成Adaptee的子类并且使得Adapter引用这个子类而不是引用Adaptee本身。

\end{itemize}

使用类适配器需要考虑的其他因素:

\begin{itemize}

	\item How much adapting does Adapter do?

	Adapter的匹配度。

	对Adaptee的接口与Target的接口进行匹配的工作量,各个Adapter可能不一样。工作范围可能是从简单的接口转换(如改变操作名)到支持完全不同的操作集合。Adapter的工作量取决于Target接口与Adaptee接口的相似程度。

	\item Pluggable adapters.

	可插入的Adapters。

	当其它的类使用一个类时,如果所需的假定条件越少,这个类就越有可复用性。如果将接口匹配构建为一个类,你就不需要假设其它类也能看到同样的接口。也就是说,接口匹配使得我们可以将自己的类加到一些现有的系统中去,而这些系统对这个类的接口可能会有所不同。

	\item Using two-way adapters to provide transparency.

	使用双向适配器提供透明操作。

	使用适配器的一个潜在问题是,它们不对所有的客户都透明。被适配的对象不再兼容Adaptee的接口,所以它们不能被Adaptee对象到处使用。双向适配器提供了这样的透明性。在不同的客户需要用不同的方法查看一个对象时,双向适配器尤其有用。



\end{itemize}


